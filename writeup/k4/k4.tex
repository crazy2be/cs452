\documentclass[titlepage]{article}
\usepackage{listings}
\usepackage{hyperref}

\lstset{basicstyle=\footnotesize\ttfamily,breaklines=true}

\begin{document}
\title{Kernel 4}
\author{Justin McGirr (\#20413625), Peter Raboud (\#20437716)}
\maketitle

\section{Instructions}
\input{"README.tex"}

\section{Design Decisions}
\subsection{Structure}
The kernel is essentially structured as a single main loop.
The loop calls the priority task queue in order to find the next ready task
to execute.
It uses the context switch to yield control flow to this task, as well as to
return control flow to the kernel when the task causes a software interrupt.
When the task returns, and sends a syscall to the kernel, the loop calls
the appropriate syscall handler in order to perform the action requested by
the syscall.
The loop continues running until all tasks exit.

Functionality is abstracted into multiple modules, listed here:

\begin{itemize}
\item A priority task queue
\item I/O subroutines
\item A context switch
\item Handlers and callers for each syscall
\end{itemize}

There are also some miscellaneous code used for initializing the kernel's environment
(such as exception handlers).

The kernel maintains some degree of separation from the user code.
User code is structured within its own directory.
By convention, it only includes headers from the \texttt{include} directory,
which limits the size of the interface between the user and kernel code.
Currently, this interface comprises the syscalls, and a method to start the
kernel from the program's entrypoint.

\subsection{Priority Queue}
The priority queue used to implement the ready queue of the scheduler is
arguably the most complex data structure used in the kernel.
The data structure currently being used is a series of FIFO queues, one for
each priority level.
When a task is enqueued, it is added to the queue corresponding to its priority
level.
When a consuming a task from the priority queue, a task is dequeued from the
non-empty queue with the highest priority.
Each individual queue is implemented as a doubly-linked list, where the
\texttt{next} and \texttt{prev} pointers of each node are part of the task
descriptor itself.
Doing this is possible since a task is in at most one queue at any given time.
Inlining these fields allows us to not need to separately allocate memory for
linked list nodes, which simplifies the design.
For efficiency, we maintain a bit-vector which allows us to efficiently find
the non-empty queue with the highest priority.
The $i$th bit of the bit vector is set if the queue of priority $i$ is non-empty.
To find the highest priority non-empty queue, can find the index of the least
significant set bit, which can be done fairly efficiently.

This implementation is reasonably efficient in both memory and time.
The memory usage is linear in the number of task descriptors, no matter
how many queues need to be created.
Moreover, the time taken to enqueue and dequeue task descriptors is constant,
and fairly fast in practice.

To be fair, this implementation is almost exactly what was suggested in the lecture.
This is by no means novel work.

We did consider alternate implementations at various points, and eventually rejected them.
The first implementation done was a priority queue implemented as a fixed-capacity heap.
However, enqueueing and dequeueing elements from the heap takes $O(\log(n))$ time,
which is too slow for our purposes.

Another implementation considered was a ringbuffer storing pointers to task descriptors.
However, we then need to have multiple ringbuffers, one for each priority level.
Each buffer needs to have sufficient space allocated to accommodate the maximum
number of tasks.
Therefore, this implementation consumed $O(kn)$ memory, where $k$ is the number
of priority levels, and $n$ is the number of tasks.
This would have gotten worse yet after implementing message passing, where each
a queue is needed for each task.
At this point, memory consumption would be $O(n^2)$, which is simply not scalable.

\subsection{Integer Log}
As a subproblem of implementing the priority queue, we needed to be able to take
a bit vector $b$, and quickly find the least significant set bit.
This allows us to quickly determine the right constituent queue to dequeue the next
task from.
To do this, we first calculate \texttt{c = b \& -b}, which leaves only the least significant
bit of $b$ set.

After that point, we want to determine the index of the set bit, which is the
base-2 log of the number.
The method used is inspired by John Owens' method as listed on the Stanford bit
twiddling hacks page, but has been altered in implementation to take fewer
instructions on the ARM processor.
The algorithm is essentially a bit-level \textbf{binary search} through the bit
vector for the set bit.

At each step, we know that the set bit is in the bottom $2i$ bits of $c$, and we want
to narrow that down to a range of $i$ bits.
At each step, we run the following pseudo-code:

\begin{center}
\begin{lstlisting}
if (c >> i) {
    c = c >> i;
    log += i;
}
\end{lstlisting}
\end{center}

If the set bit is in the high $i$ bits, then we divide by $2^i$, and add $i$ to the log.
If the set bit is in the low $i$ bits, we do nothing.
We then iterate to find the log of the new (or possibly unchanged) value of $c$.
At the end of this step, we know that set bit is within the bottom $i$ bits of $c$.
We perform this process for $i=16$, 8 and 4.

A previous copy of the algorithm then repeated the process for a shift of 2 bits and 1 bit
to find the final result.
However, it turns out that there is a trick that can be used for the last 4 cases.

Once $2i=4$, 4 values of $c$ are possible: 8, 4, 2 and 1.
The integer logs of these are 3, 2, 1 and 0 respectively.
It turns out that this is equal to \texttt{c / 2 - c / 8},
assuming integer division.
Since these divisions can be translated into bit-shifts, and because
ARM allows the second operand to many instructions to be bit-shifted essentially
for free, it turns out that this is a very cheap expression to calculate.
This trick can be used to step the last two iterations (each of which costs 3
instructions), and replace it with an addition and subtraction operation,
thereby whittling 4 instruction off the total.

The entire algorithm takes 14 instructions of hand-written assembly.

We consider the result to be undefined if $b=0$, and expect the caller to check
this case before running this algorithm to try to find the least significant set bit.
The $b=0$ case is special to the caller anyway, and needs to be handled separately,
and doing this makes it easier to optimize the algorithm.

\subsection{Destroy}
We implemented the destroy() primitive (actually, it's just the same as exit()).
This allows us to create temporary tasks, which can be a convenient way to
express many parts of the system, such as in the train server. The details of
Destroy()'s implementation are discussed in the resource allocation section,
below.

\subsection{Task Resource Allocation}
All task descriptors are simply allocated in a statically allocated table.
A task ID, modulo \texttt{NUM\_TASK}, is just an integer index into that table.
This approach makes the implementation very simple, very fast, and still allows
the reuse of task descriptors after they are destroyed (just increment the tid
for a given slot by 256 every time it is reused).

We also have a queue of free tds from which new tasks can be allocated,
allowing O(1) task creation and destruction. Although this is not extremely
important at this point, since tasks are created fairly infrequently, it does
not cost much in terms of resources or complexity (we already had a td queue
from the scheduler), and it opens up lots of possibilities for cheaply using
ephemeral tasks when it is a convenient way to express things.

Memory allocated for task stacks is statically allocated in 64K blocks. This
is almost certainly at least a couple orders of magnitude larger than it needs
to be for most tasks, but we have lots of memory currently, and using memory
for stacks is cheaper than pulling hairs when they overflow.
Like the task descriptors themselves, stacks are reused after the task exits.

\subsection{Argument Passing}
To pass arguments and return values to and from syscalls, the kernel manipulates
the context of the task as stored in memory.
During the context switch, all the registers of the user task are stored on the
user's stack, including \texttt{r0}-\texttt{r3}, which are used to pass arguments.
To find the arguments passed to the syscall, the kernel can just read the values
from the saved context.
To return a value to the user task, the kernel can write to the stored value of
\texttt{r0}, which will be restored to the user task when context switching back,
and will be interpreted as a return argument.

This approach has two advantages.
Firstly, returning values this way makes it very easy to handle returning values
to tasks which aren't immediately returned to.
The kernel just stores the return value in the context, and doesn't need to care
when that context is restored.
Secondly, it makes the context switching code very simple, since no special treatment
of the arguments is required.
The context switching simply preserves the entire context, and is agnostic to what,
if any, arguments have been passed to the syscall.

However, this approach does result in unnecessary memory accesses.
The obvious way to avoid this would be to pass the syscall arguments back to the kernel
in their original registers.
As it turns out, trying to make a C function have returns in registers (i.e.:\ a 4-member
struct returned in \texttt{r0}-\texttt{r3}) is not easy to do.
Some compilers implement non-standard extensions which support doing this, but GCC
is not one of them.
There is a plausible workaround which uses explicit register variables.
The function would return \texttt{void}, but then return values could be picked up
by accessing variables declared as being stored in \texttt{r0}-\texttt{r3}.
This would allow us to not save those registers in memory for syscalls, since
there is no expectation that those values will be preserved through the function call.
They just need to be passed to the kernel, not restored when the context switches back.
We only need to preserve the full context during interrupts, which is
currently out of scope.
However, I've had considerable difficulty in my past experiences with trying to
convince GCC to emit the correct assembly code to not clobber values returned in this
way.
Therefore, I'm not going to try to implement this solution unless it becomes a
performance problem.

\subsection{Message Passing}
Message passing is implemented by having each task keep a queue
of tasks that are waiting for replies from it (i.e. blocked). This allows us
to ensure that messages are received in FIFO order, and quickly and efficiently
find tasks that are blocked on a given task (i.e. without having to scan the
list of tasks).

In addition to this, we keep track of the current "state" of a task in it's
task descriptor, in order to ensure that invalid sequences of operations are
caught (i.e. trying to reply to a task which is not waiting for a reply).

Our message passing routines are quite fast - Around 14us in the best case for
a round trip send/receive/reply set, including context switching.

\subsection{Await Interface}
The implementation of await was something that we debated for quite some time,
because we were given multiple choices in implementation, and because there is
a lot of literature on the different approaches to device driver isolation.
Given that we are writing a microkernel, we want, for various reasons, the
kernel itself to be as small as possible. This allows isolation of failures,
easy prioritization of tasks, and more. However, given that we are, in our toy
kernel, only writing device drivers for a \emph{very} small number of
peripherals, creating all the primitives to allow safe and race-free access
to peripherals from driver tasks seemed like it would likely end up being
bigger and more complex than just implementing a HAL in the kernel. For
example, if the user task is responsible for asking the device to deassert
its interrupt, then you need to make sure that you don't enable interrupts
when switching back to the user task, or else the user task will just get
immediately interrupted again, leading to an infinite loop. However, if you
don't enable interrupts, then you can't preempt the user task, losing a
significant advantage of implementing that portion of the device communication
in userland. Thus, at least for this iteration of the kernel, we have chosen
to have the kernel read the data from the device, deassert the interrupt,
and return the relevant data back to the userland \texttt{await()} call.

The interface to \texttt{await()} is \texttt{int await(int evt, char *buf, int buflen);}.
This accommodates passing in additional data to the kernel, to support UART write events.
For timer events, no buffer is needed, or passed in.

Originally, we did not pass in a data buffer, and were considering masking this data
into \texttt{evt}.
Since \texttt{evt} is 32 bits wide, and we only need 3 to represent 5 distinct events,
we could stuff an 8-bit character into the upper bits of the \texttt{int}.
However, if we have control over the signature of \texttt{await()}, it's easier to just
add an extra parameter to contain this data.

\subsection{Await Data Structures}
To keep track of tasks that are blocked awaiting external events, for each event,
we maintain a pointer to the task descriptor blocked on that event, if any.
This only allows for a single task to wait on a given event, but this is all
we are using in our kernel.
Originally, we maintained a queue of tasks for each event, but found
that we were not using this capability, so we removed it.

\subsection{Clockserver}
As part of this assignment, we implemented a clock server as a user task, which
allows other user tasks to request the current time, or schedule themselves to run
after some delay.
I assume this will be very useful
when we start having trains going around the track, and we will start having
tasks which care about real-world values like time and distance.

The clockserver runs at \texttt{LOWER(PRIORITY\_MAX, 1)}, with the only task
that runs at higher priority being the clockserver notifier, which runs at
\texttt{PRIORITY\_MAX}. This ensures we will almost never miss ticks, unless
something particularly evil happens.

To keep track of tasks which requested future wakeup, we need a priority queue
of tasks, ordered by the time at which they need to be woken up.
We chose to use a min-heap for this purpose.
A min heap is reasonably suitable, since there aren't hard bounds on
the keys which the tasks are sorted by in the priority queue.
This made the priority-queue implementation used for the ready queue inappropriate,
since it relies on the priorities being integers in a very small range.

A min-heap allows $\log(n)$ time insertion and removal of elements, with
constant-time querying of the smallest element.
The latter is the most important property, because we have to check the top
element of this queue against the number of ticks which have passed after each tick.
The $\log(n)$ cost to add and remove elements is probably more expensive than it
sounds, since $n$ is relatively small.
However, since this is not currently a performance bottleneck, we chose not to spend
additional time optimizing for a problem which doesn't currently exist.

An additional reason for the choice of this data structure was that we coincidentally
had one from K1.
We had initially thought we might use it for the priority queue,
but instead chose an alternative data structure for that assignment.

\subsection{Interrupt-Triggered Context Switch}
As part of this assignment, we had to write a context switching
routine to be run when an interrupt was triggered.
We chose to implement this as a context switch into the kernel, where the interface to the kernel just looks
like an SVC call, but with a \texttt{SYSCALL\_IRQ} syscall number, where the kernel then
handles the interrupt just as it would a regular syscall. We considered having
an extra pseudo-task to handle interrupts, with it's own stack, which would
automatically get ``scheduled'' whenever an interrupt was triggered, but decided
that the additional complexity would not be worth the benefits, and additionally
such a design would waste some space --- the IRQ stack would only be used when
IRQs are triggered, and is unlikely to hold much of anything useful the rest
of the time.

The actual context switching function is different depending on whether the
interrupt originated from an SVC call or a IRQ, although almost all of the
source code is shared through the use of assembly macros. Although we considered
trying to fit both context switches into a single block of code through the use
of conditional execution, it seemed like it would complicate things
considerably, not least of which because you would have to find a register to
store that state in. As the context switch is only slightly more than ten
instructions, we opted just to implement the conditions at the assembler level.

% We should talk about:
% What tasks we have in the kernel, what they do, and why.
% What priority each of the tasks runs at.
% How we do kernel debugging (i.e. bwio and how it just works even though the
% assignment description says that it shouldn't).

% what we should talk about
% commandsrv, trainsrv, sensorsrv
% iosrv
% interrupts for io, cts, single-await, fifos
% shutdown
% destroy

\subsection{IO in the Kernel}
The syscall interface we chose to have for IO is for user tasks to pass in a buffer
filled with data they want to output, or a buffer they wish to have filled with input.
They also pass an event corresponding to the COM port they want to do IO with.
The kernel runs code in response to hardware interrupts to actually do the IO.
This is pushes most of the hardware-specific IO logic into the kernel, as opposed to
the alternate approach where the kernel merely signals to the user task that it is possible
to do IO, and leaves it to the user task to actually perform the IO.

This was done mostly based on the belief that the IO should be entirely done in the kernel,
with the hope that the IO server can eventually be subsumed into the kernel entirely.
The userland implementation of the IO server is fairly complex, and much of the logic
in it is related to message passing boilerplate.
Since this could be eliminated in the kernel, moving all of the IO buffering logic
into the kernel could allow us to simply eliminate this boilerplate, and simplify the code.
It is unclear, however, if we will ever have sufficient time to justify actually doing
this, since the IO server seems to not be causing any problems at the present.

The way that the IO interrupts work is that when a task begins awaiting on a particular
transmit or receive event, we set the control register in the UART to assert an interrupt
when that event occurs.
We then leave the kernel, and wait for the event to occur.
(A simple optimization we could do is to check if the event has already happened by
inspecting the UART registers, which would allow us to avoid a context switch.)
When the interrupt occurs, for transmit events we then perform writes to the UART,
until there is no more data to be transmitted, or until the UART can't accept any more.
If there is data left to be transmitted, we exit the kernel again, and wait for the TX
interrupt to be asserted again, and repeat this process.
If there isn't data left, we deassert the interrupt, and schedule the awaiting task
to be run.
Reads are processed in a similar way.
IO on COM2 uses the FIFOs, which is conceptually no different - it simply allows
us to process multiple bytes of data each cycle, instead of a single one.

We changed the signature of the await syscall to allow passing in a buffer of data,
wheras previous signature only allowed passing in a single byte.
We needed to make this change when we started using the hardware FIFOs in the UARTs.
The point of using the FIFOs is that we can check the UARTs less frequently, and have
less overhead per byte of data transferred.
If we need to do a syscall (meaning a pair of context switches) for each byte anyway,
there is absolutely no advantage to using the FIFOs.
Therefore, we need to pass in buffers to take advantage of this.

In order to transmit reliably to the train controller, we needed to transmit slowly
enough to not flood the train controller.
We use the CTS bit asserted by the train controller to do this.
After transmitting, we wait for a modem interrupt to signal CTS being deasserted, then
asserted again, in addition to the normal condition of the TX interrupt being asserted.
We use a state machine to keep track of these transitions.
This is only done for COM1, since COM2 can process input sufficiently quickly for
this to not be a problem.
COM1 doesn't use the FIFOs, since we lack the hardware support to do CTS properly
in hardware.

\section{IO Server}
We implement an IO server in userland, which the other user tasks interact with
in order to perform IO.
There is one IO server for each serial port, which handles both input and output
for that port.
The IO server has a pair of notifiers, one to notify it of inbound data, and another
to send outbound data.
Since most of the work is done in the kernel, the only thing that the IO server
is responsible for is buffering IO in both directions.
For writes, user tasks writing to the IO server return immediately, and don't need
to wait until the data is actually flushed out to the serial port.
For reads, data is buffered in the IO server if there isn't a task which wants
that data at the time that it is read.
This prevents data from being dropped if no task is attempting to consume it at a particular moment.
(More accurately, this substantially reduces that probability, since the read notifiers
spend almost all of their time awaiting events, and never spend more than a few microseconds
at a time not waiting. We simply don't receive input at a fast enough rate for it to be a problem, since
the hardware can buffer it for the time the notifiers aren't listening.)
The IO notifiers have the second highest priority possible, second only to the timer notifier.
The IO server itself has the fourth highest priority.

\section{Halting}
We have added a \texttt{halt()} syscall into the kernel.
The kernel can no longer know when it is safe to exit from the kernel,
because the receive notifier task is problematic.
Previously, we would shutdown all of the servers from userland.
When the servers' notifiers next try to communicate with the server, they would realize
that the server exited, and they would respond by exiting as well.
Since the only notifier was the timer notifier, we could be assured that the notifier
would wake up fairly shortly, so this approach worked.
However, the receive notifier may never wake up, if no input is sent to the UART,
so this approach is not viable.
The way we handled this is to allow a user task to signal to the kernel that it
can shut down immediately.
Another way that we could have handled this is to make the receive await time out
periodically, so that we could guarantee that all notifiers exit shortly after
the user tells the program to quit.

\subsection{Train Tasks}
As required for this assignment, we implemented several user tasks to
reimplement a0. There are four user tasks used for this purpose:
\begin{description}
\item[sensorsrv] Constantly polls the sensors and feeds the updated data to
	the displaysrv.
\item[commandsrv] Constantly waits for user input, parses the input, feeds
	characters as they are typed to the displaysrv, and sends parsed commands
	to trainsrv for execution.
\item[displaysrv] Prints the track layout to the screen, including any updates
	that occur as a result of changes to the track (switches changing, sensors
	getting tripped, etc).
\item[trainsrv] Drives the trains. All requests for actions on the trains or
	the track goes through this server, including setting the speed of trains,
	actuating switches, etc. When things on the track change, this task tells
	the displaysrv that things have changed, thus ensuring that the display
	never gets out of sync. (Currently it notifies the displaysrv by way of
	a \texttt{send()}. This currently works well enough because our i/o output
	primitives are non-blocking, so displaysrv never takes very long to run,
	but we may have to replace this with a courier task if this changes in
	the future).
\end{description}
Of these, the trainsrv is the only only which is at all novel in it's
implementation. In particular, it uses ephemeral tasks to reset the solenoid
for sensors some time after they are actuated, and for properly reversing
trains without a sudden stop. This implementation reduced explicit state, and
makes the code easier to follow, although it does have some additional
overhead, and required the implementation of \texttt{destroy()}.

\section{Source Code}
The source code is hosted on git, at \url{git.uwaterloo.ca/pgraboud/cs452-kernel}.
The version we wish to submit is on the \texttt{k4} branch, specifically
the commit:
\input{|"git rev-parse k4 | ./../verbatim"}
We are submitting the following files:
\input{|"cat ../file-list | ./../verbatim"}
\input{|"./../file-list | ./../verbatim"}

\end{document}
