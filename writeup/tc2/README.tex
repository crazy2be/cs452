\subsection{Installation}\label{installation}

On the student cs environment, simply run
\texttt{make -j8 ENV=ts7200 TYPE=t install}. This will place the
compiled kernel in \texttt{/u/cs452/tftp/ARM/\$(USER)/t.elf} (where
\texttt{\$(USER)} is whatever your username is on the computer which you
are compiling it). You can then load this kernel from redboot using
\texttt{load -b 0x00218000 -h 10.15.167.5 "ARM/\$(USER)/t.elf"}.

Note that this depends on a toolchain installed in my home directory,
\texttt{/u1/jmcgirr/arm-toolchain}. This is currently world-readable, so
it should work regardless of who is trying to run the Makefile, but
please do let us know if you encounter any issues!

\subsection{Emulator}\label{emulator}

You can run the OS on an emulated arm CPU via
\texttt{make -j8 ENV=qemu TYPE=t qemu-run}. In order for this to
complete successfully, you will need:

\begin{itemize}
\itemsep1pt\parskip0pt\parsep0pt
\item
  A local install of QEMU
\item
  A local install of an arm-none-eabi toolchain
\end{itemize}

\subsection{Tests}\label{tests}

The provided kernel has some basic tests, which you can run with
\texttt{make qemu-test}. (this depends on the emulator, above).

\subsection{Debugging}\label{debugging}

If something goes wrong and you want to investigate deeper, you can run
\texttt{make qemu-start} in one terminal (this starts qemu in a
suspended state), and then \texttt{make qemu-debug} in another. You can
then set breakpoints, step, inspect memory, etc- anything you can
normally do with gdb.

A simulator for the track itself is also available in
\texttt{src/mock-train/main.py}. It requires installing
\href{https://graph-tool.skewed.de/}{graph-tool}. To run it, run

\begin{verbatim}
make qemu-run

# in another window
python src/mock-train/main.py

# in a third window
telnet localhost 1231
\end{verbatim}

This display a graphical view of the simulated train set. The simulator
communicates with the emulated program with the same binary protocol as
the real train controller (although it processes input faster), so that
it's transparent to the program whether it's talking to the real train
set, or the simulated one.

The simulator is by no means perfect, and there are many failure cases
which aren't reproducible on the simulator. The simulated trains have
different acceleration, deceleration, and velocity characteristics than
the real trains. In particular, the simulated trains aren't affected by
track conditions, so the velocity of the train doesn't vary across
sections of track. Also, trains are treated as being infinitesimally
small points, instead of having size. This allows performing maneuvers
in the simulated trains which would result in a crash on the real track.

Despite its limitations, the simulator is an extremely valuable tool for
rapidly debugging routing or track reservation issues, because it's
faster to iterate on a simulator than to load and run the code on the
real hardware. The advantage is particularly striking when the track is
heavily contended.

\subsection{Commands}\label{commands}

\begin{itemize}
\itemsep1pt\parskip0pt\parsep0pt
\item
  \texttt{tr \textless{}train number\textgreater{} \textless{}speed\textgreater{}}
  sets the speed of a train.
\item
  \texttt{sw \textless{}switch number\textgreater{} (c\textbar{}s)} sets
  a the position of a switch to curved or straight.
\item
  \texttt{bsw (c\textbar{}s)} bulk set all switches to the given
  configuration (curved or straight).
\item
  \texttt{rv \textless{}train number\textgreater{}} reverses a train.
\item
  \texttt{stop \textless{}train number\textgreater{} \textless{}node name\textgreater{} {[}\textless{}edge\textgreater{}{]} \textless{}displacement\textgreater{}}
  stops the train at the given node, or offset thereof. I.e.
  \texttt{stp 63 E8 0} would stop train 63 right on top of sensor E8. If
  \texttt{\textless{}edge\textgreater{}} is specified, it denotes which
  edge the offset is along (i.e.~for a branch node).
\item
  \texttt{bisect \textless{}train number\textgreater{}} performs
  bisection to calibrate the stopping distance for
  \texttt{\textless{}train number\textgreater{}}, exiting automatically
  after 5 iterations.
\item
  \texttt{route \textless{}train number\textgreater{} \textless{}node name\textgreater{}}
  routes the given train to a particular node. Currently, the train must
  be fully stopped for the routing to work properly. It's also necessary
  to run the train manually until it hits a sensor, so that the program
  picks up the position of the train.
\item
  \texttt{f} freezes the terminal output until typed again. This is
  useful for copying log data out of the terminal program without it
  being overwritten.
\item
  \texttt{q} exits the program.
\end{itemize}

The positions of each switch is shown in the ASCII-art map of the train
track. When a sensor is tripped, it is shown both on the ASCII-art map
for as long as it is tripped, as well as to the right, on the list of
recently-fired sensors.

Because sensor attribution is fault-tolerant, the program will recover
if a single switch fails or a sensor fails to register. After the train
hits the next switch, the program will recognize the train's new
position, and correctly display it. Furthermore, if a switch failed, the
program will correctly recognize that the switch is in the opposite
orientation it thought it was in, and update the internal model. The
program is also tolerant of spurious sensor hits, and will ignore sensor
hits that it thinks weren't caused by a train.
