\subsection{Installation}\label{installation}

On the student cs environment, simply run
\texttt{make ENV=ts7200 KIND=t install}. This will place the compiled
kernel in \texttt{/u/cs452/tftp/ARM/\$(USER)/t.elf} (where
\texttt{\$(USER)} is whatever your username is on the computer which you
are compiling it). You can then load this kernel from redboot using
\texttt{load -b 0x00218000 -h 10.15.167.5 "ARM/\$(USER)/t.elf"}.

Note that this depends on a toolchain installed in my home directory,
\texttt{/u1/jmcgirr/arm-toolchain}. This is curretly world-readable, so
it should work regardless of who is trying to run the Makefile, but
please do let us know if you encounter any issues!

\subsection{Emulator}\label{emulator}

You can run the OS on an emulated arm CPU via
\texttt{make ENV=qemu KIND=t qemu-run}. In order for this to complete
successfully, you will need:

\begin{itemize}
\itemsep1pt\parskip0pt\parsep0pt
\item
  A local install of QEMU
\item
  A local install of an arm-none-eabi toolchain
\end{itemize}

\subsection{Tests}\label{tests}

The provided kernel has some basic tests, which you can run with
\texttt{make qemu-test}. (this depends on the emulator, above).

\subsection{Debugging}\label{debugging}

If something goes wrong and you want to investigate deeper, you can run
\texttt{make qemu-start} in one terminal (this starts qemu in a
suspended state), and then \texttt{make qemu-debug} in another. You can
then set breakpoints, step, inspect memory, etc- anything you can
normally do with gdb.

\subsection{Special Instructions}\label{special-instructions}

Before running the program on the TS7200, it is necessary to reset both
the TS7200 and the train controller, to avoid messy state left behind by
previous runs. In particular, this has been known to cause bugs with the
sensor reads, where garbage data left in the UART is picked up, and
misinterpreted.

\subsection{Commands}\label{commands}

\begin{itemize}
\itemsep1pt\parskip0pt\parsep0pt
\item
  \texttt{tr \textless{}train number\textgreater{} \textless{}speed\textgreater{}}
  sets the speed of a train.
\item
  \texttt{sw \textless{}switch number\textgreater{} (c\textbar{}s)} sets
  a the position of a switch to curved or straight.
\item
  \texttt{bsw (c\textbar{}s)} bulk set all switches to the given
  configuration (curved or straight).
\item
  \texttt{rv \textless{}train number\textgreater{}} reverses a train.
\item
  \texttt{stp \textless{}train number\textgreater{} \textless{}node name\textgreater{} {[}\textless{}edge\textgreater{}{]} \textless{}displacement\textgreater{}}
  stops the train at the given node, or offset thereof. I.e.
  \texttt{stp 63 E8 0} would stop train 63 right on top of sensor E8. If
  \texttt{\textless{}edge\textgreater{}} is specified, it denotes which
  edge the offset is along (i.e.~for a branch node).
\item
  \texttt{bisect \textless{}train number\textgreater{}} performs
  bisection to calibrate the stopping distance for
  \texttt{\textless{}train number\textgreater{}}, exiting automatically
  after 5 iterations.
\item
  \texttt{q} exits the program.
\end{itemize}

The positions of each switch is shown in the ASCII-art map of the train
track. When a sensor is tripped, it is shown both on the ASCII-art map
for as long as it is tripped, as well as to the right, on the list of
recently-fired sensors.
