\subsection{Installation}\label{installation}

Simply run \texttt{make ENV=ts7200 install}. This will place the
compiled kernel in \texttt{/u/cs452/tftp/ARM/\$(USER)/k.elf} (where
\texttt{\$(USER)} is whatever your username is on the computer which you
are compiling it). You can then load this kernel from redboot using
\texttt{load -b 0x00218000 -h 10.15.167.5 "ARM/\$(USER)/k.elf"}. At this
stage, the kernel doesn't do much that is particularilly interesting,
there is no shell or other interactive elements, it simply runs the code
which it was given.

\subsection{Emulator}\label{emulator}

You can run the OS on an emulated arm CPU via
\texttt{make ENV=qemu qemu-run}. In order for this to complete
successfully, you will need:

\begin{itemize}
\itemsep1pt\parskip0pt\parsep0pt
\item
  A local install of QEMU
\item
  A local install of an arm-none-eabi toolchain
\end{itemize}

\subsection{Tests}\label{tests}

The provided kernel has some basic tests, which you can run with
\texttt{make ENV=qemu test}. (this depends on the emulator, above).

\subsection{Debugging}\label{debugging}

If something goes wrong and you want to investigate deeper, you can run
\texttt{make ENV=qemu qemu-start} in one terminal (this starts qemu in a
suspended state), and then \texttt{make ENV=qemu qemu-debug} in another.
You can then set breakpoints, step, inspect memory, etc- anything you
can normally do with gdb.
